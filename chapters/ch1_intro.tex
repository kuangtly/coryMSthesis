\chapter{Introduction} 

When you got a new camera, you probably will take a lot of testing photos before you start to use it seriously. 

Hardware injection test is a process to verify the performance of interferometer by sending real sample signal into interferometer. Ideally, we should prepare some real gravitational waves as test signal. But it is practically hard to generate large enough artificial gravitational waves that are detectable by current technology. Therefore, as an injection signal, people shake the mirror according to the waveform template, thereby changing arm length correspondingly. 

\section{Introduction to Gravitational Wave}
\subsection{What is gravitational wave}

In the General Theory of Relativity proposed by Albert Einstein in 1915, phenomena caused by gravity can be interpreted as results of curved spacetime. This is one of his important work after his `Happiest Thought', which recorded in his unpublished article ``Fundamental Ideas and Methods of the Theory of Relativity, Presented in Their Development"\cite{Einstein:happy}. Among different ways to curve our spacetime, which can be described by corresponding metric tensor fields, there exist wavelike solutions describing ripples of spacetime known as gravitational waves.

However, the physical reality of gravitational wave is not so clear to everyone in the early days, even to Even Einstein himself ~\cite{Einstein:prl, Einstein:ongw}. The main problems is that there exist some gauge degree of freedom in the theory due to the arbitrariness of coordinate choices. We have to know whether the gravitational waves we found are just gauge waves (vibration of coordinate) or the wave can have some observable consequences. 

One of the most important observational evidence implying the existence of real gravitational waves is Hulse-Taylor pulsar~\cite{HulseTaylor:discovery}. Taylor demonstrated that the change of pulsar rotation speed can be explained by emission of gravitational wave\cite{HulseTaylor:gr}.  
....
Finally, in 2015 September 14.
GW150914

\subsection{How to describe gravitational wave}

In Einstein's General Relativity, gravitational effects are realized by geometry of spacetime. According to great mathematician Bernhard Riemann, we can describe the geometry of certain space by telling the "distance" between nearby points in the space. Practically, the information of distance between nearby spacetime points form a tensor called metric, which means the measure of distance. If such metric tensor exist everywhere in our universe, they become metric tensor field, which tell us the shape or geometry of our universe. By choosing a coordinate system, one can write down those corresponding components $g_{\mu\nu}$ of metric tensor g. Now, we can calculate spacetime distance$ds$ between two nearby points by their coordinate separation:

\begin{equation}
    ds^2 = g_{\mu\nu} dx^{\mu} dx^{\nu}
\end{equation}




\begin{equation}
    \Box \bar{h}_{\mu\nu} = - \frac{16\pi G}{c^4}T_{\mu\nu}
\end{equation}


mathematics….

TTgauge
There is a very convenient gauge(coordinate) constrain, which is known as Traceless and Transverse gauge. 
for perturbation of wave like part metic over Schwarzschild background which represent Earth’s gravity.


Refer to \cite{maggiore:gw1}



How to generate gravitational wave
The source of electromagnetic wave is time-dependent electrical charge distribution.  Similarly, the source of gravitational wave is time-dependent mass(energy) distribution. Strictly speaking, the lowest order of mass multi-pole which can generate real gravitational waves is mass quadruple because we don’t have negative mass, while the electromagnet wave can be generate though time-dependent electrical dipole moment. The gravitational wave strain generated by mass quadruple can be approximately described by famous quadrupole formula:
(quadrupole formula)
Practically, PN NR....
According to our current understand of universe, there are several kinds of astrophysical gravitational wave sources, whose h(t) amplitude is large enough to be detected by current ground based laser interferometer, like advanced-LIGO, advanced-Virgo or KAGRA. 
Compact Binary Coalescence 
BNS BBH
\subsection{How to detect Gravitational wave}
The interaction of detector and gravitational wave can have different interpretation due to different coordinate choice\cite{ifo:gauge}. It is quiet similar to that the magnetic force in one observational frame may be electric force in the other frame. However, practically, I would like to use the …….., which described in next section.


\section{Detection of Gravitation wave}

Interaction of GW wave when $\lambda_gw$ L of detector
Limit of Michelson IFO
IFO with dual-recycling and Fabry-Perot  arms.
 Complex response
WE NEED Calibration Calibration Calibration



\section{Calibration and Reconstruction}

Calibration is always the fist step before we measure something by some device.
For example, to measure the weight of an apple, you should calibrate your scale by putting a “standard kilogram” on it. Then, you can either adjust the scale readout to be 1kg, or record the difference showed in scale readout, which may be used to reconstruct real weight of the apple. However, the spring constant of springs inside the scale could fluctuate due to temperature changes. To accurately measure the weight of the apple, we have to measure the calibration factor (scale readout when we put the standard mass on it.) when we measure the weight of apple, if possible, simultaneously.

Due to the complexity of practical interferometer, the response of interferometer itself to external gravitational source is not only complex but also time-dependent. In reality, we “inject several calibration lines”, which means we shake the End-Test-Mirror by several known frequency and amplitude sine wave. Then, we try to see these standard signal in readout of interferometer. If we can solve 
……. 


\subsection{Transfer function of Laser Interferometer with Fabry-Perot Cavity}
\subsection{Tracking Time-dependent Response by Calibration lines}



\section{Photon Calibrator (Pcal)}
\subsection{Principle of Photon Calibrator}
Photon calibrator is an additional laser with high precision intensity modulator. It is installed in front of End-Test-Mass Mirror(ETM) and can push the ETM by radiation force due to its own Laser beam. To generate any artificial h(t) by Pcal, we have to translate desired h(t) into corresponding force F(t) exerting on ETM. This can be done by using equation of motion of the ETM suspend by its suspension system. Then, we control the Pcal Laser output intensity P(t) such that the radiation force exerted on ETM is F(t) we calculated before. If we analyze it frequency domain,  the h(f) introduced by P(f) can be describe by eq:

Original Pcal is proposed by Glasgow group \cite{pcal:clubley2001}. They use single laser beam hitting on the center of ETM. The problem is that it may introduce drumhead mode vibration of ETM surface (just like the vibration mode you see when you hit the center of a drum), which introduce unwanted h(t) effectively. This problem is solved by LIGO group\cite{pcal:karki2016}, who separate the Pcal laser beam into two beams, hitting on the nodal point of drumhead mode on the ETM surface\cite{pcal:Daveloza2012}.  

 —> KAGRA

In order to excite same amplitude h(t) in higher frequency regime, we have to give much larger F(t) since the relationship between x(t) and F(t) in an pendulum ……………….



\subsection{Why do we need Photon Calibrator}
\subsection{Tracking Time-dependent Response by Calibration lines}
