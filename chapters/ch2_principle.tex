
\chapter{Hardware Injection through Photon Calibrator}
\section{Principle}
 
 
 

%
%Validate IFO by Pcal (Hardware Injection Test)
%As I mentioned in last chapter, the practical response of IFO is very complex. To prevent some unexpected problem including no-linear response of IFO and……  
%The best way is to provide some test source of expected GW signal.
%
%However, it is almost impossible to prepare, for example, a binary blackhole system in laboratory. Instead, we will generated some test signal by pushing the ETM with Pcal. This procedure is called “Hardware Injection Test”
%
%
%Motivation
%To under whether we can successfully reconstruct the h(t) from our interferometer, the best way is to prepare an artificial signal, sending it to interferometer, reconstructing it, finally, comparing it with original one. However it is quite difficult to generate human made gravitational wave that can be detected by current gravitational wave detector.
%
%Requirement
%
%Low Frequency
%around 100Hz  the nose should below the IFO sensitivity
%(absolute timing < ?us ns )
%
%High Frequency
%above 1kHz    the transfer function should as flat as possible


\subsubsection{Amplitude of Injection Signal}

%\begin{equation}
%\label{eq:pcaleq}
%    \Delta L (f) (\mathrm{m} / \mathrm{Hz}) = \frac{2 \Delta P \cos(\theta)}{c} \frac{1}{M (2 \pi f )^2}
%\end{equation}

\begin{align}
%\label{eq:pcaleq}
    \frac{F(t)}{M}=\frac{1}{M} \frac{2 P(t) \cos(\theta)}{c} &= \ddot{x}(t)
\end{align}

For $x=x_0 \sin(\omega t)$,
\begin{align}
%\label{eq:pcaleq}
    \frac{1}{M} \frac{2 P_0 \cos(\theta)}{c} \sin(\omega t) &=  -\omega^2 x_0 \sin(\omega t)
\end{align}

Thus,
\begin{align}
%\label{eq:pcaleq}
    P_0 &= -\omega^2 \frac{M c}{2 \cos(\theta)} x_0 = -\omega^2 \frac{M c}{2 \cos(\theta)} L h_0
\end{align}
\begin{align*}
%\label{eq:pcaleq}
    M &= 23 ~\mathrm{kg} \\
    L &= 3 ~\mathrm{km}  \\
    \theta &= 0.72 ~\mathrm{deg}  \\
    c &= 2.998\times10^8 ~\mathrm{m/s} \\
    P_0 ~(\mathrm{Watts}) \times \frac{Gain_{\text{~Power to OFSPD}}}{2} &= 
     \underbrace{V_{\text{OFSPD}}}_{\text{Same as V$_{\text{Injection}}$}}~ (\mathrm{Volts}) \\
\end{align*}
Therefore, the overall gain should be set in injection channel, which is in Volt unit, is
\begin{align}
%\label{eq:pcaleq}
    \omega^2 \frac{M c}{2 \cos(\theta)} L \times \frac{Gain_{\text{~Power to OFSPD}} }{2}
\end{align}

